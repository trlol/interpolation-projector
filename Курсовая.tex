\documentclass[14pt,openany,a4paper,oneside]{extarticle}
\usepackage{mmap}
\usepackage{cmap}
\usepackage[T2A]{fontenc}
\usepackage[utf8]{inputenc}
\usepackage[english, russian]{babel} 
\usepackage{indentfirst}
\usepackage{graphicx}
\usepackage{amsmath,amsfonts,amssymb,amscd,amsthm}
\usepackage[unicode, pdftex]{hyperref}
\linespread{1}
\setlength{\parindent}{1.25 cm}
\usepackage[a-1b,usecharset]{pdfx}
\usepackage[a4paper, mag=1000, left=3cm, right=2cm, top=2cm, bottom=2cm, headsep=0.7cm, footskip=1cm]{geometry}
\pagestyle{plain}
\usepackage{wrapfig}
\usepackage{mathrsfs}
\usepackage{mathtext}
\usepackage{epsf}
\usepackage{cite}
\usepackage{fancyhdr}
\usepackage{listings}
\usepackage{array}
\usepackage{ifpdf}
\usepackage{tocvsec2}
\usepackage[normalem]{ulem} 
\usepackage{setspace}
\DeclareMathOperator{\tr}{tr}
\sloppy
\setcounter{tocdepth}{4}
\raggedbottom
\setlength{\parskip}{\medskipamount}
\renewcommand{\baselinestretch}{1} 
\usepackage{lastpage}
\makeatletter
\renewcommand{\@makecaption}[2]{
	\vspace{\abovecaptionskip}
	\sbox{\@tempboxa}{#1. #2} \ifdim \wd\@tempboxa > \hsize #1. #2\par
	\else \global\@minipagefalse \hbox \to \hsize {\hfil #1. #2\hfil}
	\fi \vspace{\belowcaptionskip}}
\makeatother
\usepackage[labelsep=period]{caption}
\usepackage{float}
\usepackage{colortbl}
\usepackage{minted}
\usepackage{dirtree}
\captionsetup[figure]{labelfont={bf}}
\begin{document}
	\setcounter{page}{2}
	\section*{Реферат}
	\thispagestyle{empty}
	
	Объем 23 с., 2 гл., 6 рис., 1 табл., 5 источников, 1 прил.
	
	\textbf{Простые числа, криптография, генерация чисел, вероятностные тесты, алгоритм Миллера–Рабина.}
	
	Объектом исследования данной работы является задача построения больших простых чисел, необходимых для криптографических систем.
	
	Цель работы — исследование теоретико-числовых основ и практических алгоритмов генерации больших простых чисел, применяемых в современных криптографических протоколах, на примере реализации на языке Python.
	
	В рамках исследования был проведён анализ детерминированных и вероятностных методов проверки простоты чисел, в том числе алгоритма Миллера–Рабина. Практическая часть включает реализацию генератора больших простых чисел с использованием безопасной генерации случайных чисел и тестов на простоту.
	
	Результаты данной работы могут быть использованы при разработке программных решений в области криптографии, где критически важны надёжные и эффективно вычисляемые простые числа высокой разрядности.
	
	\newpage
	\def\contentsname{Содержание}
	
	\tableofcontents
	
	\newpage
	\section*{Введение}\addcontentsline{toc}{section}{Введение}
\end{document}